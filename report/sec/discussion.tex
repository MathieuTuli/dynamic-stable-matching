\section{Discussion}
In this work, we introduce the dynamic stable matching setting where each agent's preference profile is associated with an underlying numerical utilities vector, which is subject to change over time. Hence, a good matching algorithm needs to produce matchings that both maximize the social welfare and are long-lasting. Our empirical evaluations show that compared to some other algorithms, the classical stable matching algorithms produce very inconsistent matchings that also achieve relatively low social welfare with many agents across a long time horizon. Because current matching algorithms have no acknowledgement of horizon, it is expected for them exhibit such behaviour. In future work, it would interesting to investigate methodologies that are aware of the dynamics of agents and could control stability and optimality over longer horizons. Such methods are much more useful for everyday settings as real environments exhibit complex but sometimes characterizable dynamics. Further, in this work we considered simulated dynamics using fairly basic Gaussian distributions, thus it would be interesting to model more realistic dynamics based on e.g., real-world marriage statistics~\cite{lavner2010patterns}.
