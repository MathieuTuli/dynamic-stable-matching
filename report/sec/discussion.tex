\section{Discussion}
In this work we introduce the dynamic stable matching setting where each agent's preference profile is associated with an underlying numerical utilities vector, which is subject to change over time. Hence, a good matching algorithm needs to produce matching that respects both social welfare and the longevity of marriages over time. Our empirical evaluations show that the classical stable matching algorithms produce very inconsistent matching that also does not have high social welfare in settings with more agents and longer time horizon, compared to a few other matching algorithms. Because current matching algorithms have no acknoledgement of horizon, it is expected for them exhibit such behaviour. In future work, it would interesting to investigate methododlogies that are aware of the dynamics of agents and could control stability and optimality over longer horizons. Such methods are much more useful for everyday settings as real environments exhbits complex but sometimes characterizable dynamics. Further, in this work we considered simulated dynamics using faily basic Gaussian distributions, thus is would be further interesting to consider more realistic dynamics based on statistics say.
