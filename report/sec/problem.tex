\section{Problem Formulation}
% \begin{itemize}
%     \item 1.5 - 2 pages
%     \item Formulation for the set-up with utilities, excitement and dynamics
%     \item Metrics
%     \begin{itemize}
%         \item Stability
%         \item Social Welfare
%         \item Consistency
%         \item Explore social welfare vs. stability
%     \end{itemize}
% \end{itemize}
In this section, we formally describe the dynamic stable matching setting, followed by a definition of several evaluation metrics.
\subsection{Problem Setting}
In the classical stable matching problem setting, there are $N$ men and $N$ women where each man $m_i$ has a strict preference ranking $\overrightarrow{\succ}_{m_i}$ for the women and vice versa, and the goal is to find a perfect and stable matching $M$ such that no pair of man $m$ and woman $w$ prefer each other to their current matches. In this project, we study a variant of the stable matching problem with the modification that each agent's preference is induced by underlying numerical utilities, and that the utilities may change over time, prompting a need to re-match agents according to the updated preference profiles.

More formally, in our \textit{dynamic} stable matching setting, there are again $N$ men and $N$ women. Each man $m_i$ is associated with an underlying numerical utility vector $u_i \in \mathbb{R}^n$ satisfying $w_{j_1} \succ_{m_i} w_{j_2} \Leftrightarrow u_i(w_{j_1}) > u_i(w_{j_2})$ for every pair of women $(w_{j_1}, w_{j_2})$. Similarly, each women $w_j$ is associated with a utility vector $v_j \in \mathbb{R}^n$ consistent with her preferences for the group of men. The utilities are non-negative and normalized, i.e., $\sum_{j=1}^N{u_i(w_j)} = 1$ for every $m_i$, and $\sum_{i=1}^N{v_j(m_i)} = 1$ for every $w_j$.

In addition, each agent has an inherent excitement value towards each viable candidate. Formally, each man $m_i$ has an excitement vector $x_i \in \mathbb{R}^n$ where $x_i(w_j)$ denotes the excitement man $m_i$ has for woman $w_j$. Similarly, each woman $w_j$ has an excitement vector $y_j \in \mathbb{R}^n$ representing her excitement towards each man. All excitement values are non-negative.

Furthermore, the dynamic stable matching setting adds a time dimension. For simplicity, instead of working with the infinite continuous time domain, we define a finite set of discrete times $t \in \{1, \ldots, T\}$, and each agent's utility vector changes over time according to the following rule. At each time step $t$, with everyone's utilities $\overrightarrow{u}^{t}$ and $\overrightarrow{v}^{t}$ and the current matching $M^t$, we compute the updated utilities $\overrightarrow{u}^{t+1}$ and $\overrightarrow{v}^{t+1}$ for each man $m_i$ and woman $w_j$ as follows:
    \begin{align*}
        & u^{t+1}_i(w_j) = \begin{cases} u^t_i(w_j) * (1 + x_i(w_j)) &\text{if $M^t(m_i) \neq w_j$} \\ u^t_i(w_j) * \max(0, (1 - x_i(w_j))) &\text{if $M^t(m_i) = w_j$} \end{cases},\\
        & v^{t+1}_j(m_i) = \begin{cases} v^t_j(m_i) * (1 + y_j(m_i)) &\text{if $M^t(w_j) \neq m_i$} \\ v^t_j(m_i) * \max(0, (1 - y_j(m_i))) &\text{if $M^t(w_j) = m_i$} \end{cases}.
    \end{align*}
In other words, the matched couple have decreased utilities towards each other and increased utilities for all other candidates, and the extent to which the utilities change is based on the excitement values. The update rule ensures that the new utilities $\overrightarrow{u}^{t+1}$ and $\overrightarrow{v}^{t+1}$ are non-negative, and we further normalize them. With the updated normalized utilities, we re-run the matching algorithm to generate a new matching $M^{t+1}$. Depending on the new utilities, matched couples from $M^t$ may not stay together in the new matching $M^{t+1}$.


\subsection{Metrics}
We design this new setting with inspirations from relationships in the real world, where people can divorce and enter new marriages over time. A desirable matching algorithm should both respect the agents' dynamic utility profile and produce long-lasting marriages.

\paragraph{Stability} We measure the instability of a series of matches $(M^1, \ldots, M^T)$ with the average number of blocking pairs across time, i.e.,
$
    \mbox{instability}(M^1, \ldots, M^T) = \frac{1}{T}\sum_{t=1}^T{\frac{\sigma(M^t)}{N}},
$
where $\sigma(M^t)$ is the total number of blocking pairs in $M^t$ according to the preference profile at time $t$.
\paragraph{Social Welfare} We define the social welfare of a series of matches as the mean utilities each agent derives from the matching across time:
$
    \mbox{sw}(M^1, \ldots, M^T) = \frac{1}{T}\sum_{t=1}^T{\frac{1}{2N}{\sum_{(m_i, w_j)\in M^t}{u_i(w_j) + v_j(m_i)}}}.
$
\paragraph{Consistency} We also would like to encourage the couples to stay together over time. We define the consistency of a series of matches to be:
$
    \mbox{consistency}(M^1, \ldots, M^T) = \frac{1}{T-1}\sum_{t=1}^{T-1}{|M^t \cap M^{t+1}|},
$
where $M^t \cap M^{t+1} = \{(m, w): (m, w) \in M^t \wedge (m, w) \in M^{t+1}\}$ is the set of couples that stay together in the matching between time $t$ and $t+1$.

\paragraph{Stability vs. Social Welfare} Note that stability and social welfare are both metrics that take into account of the preference profiles/utilities of the agents. They are in generally correlated, but a max social welfare matching $M$ is not necessarily stable, and a stable matching might not maximize social welfare.

Max social welfare $\nRightarrow$ stability: Consider two men and two women where $m_1$ has utility 1 towards $w_1$ and utility 0 towards $w_2$, $m_2$ has utility 0.6 towards $w_1$ and 0.4 towards $w_2$, $w_1$ has utility 0.4 towards $m_1$ and 0.6 towards $m_2$, and $w_2$ has utility 0 towards $m_1$ and 1 towards $m_2$. In this setting, $M = \{(w_1, m_1), (w_2, m_2)\}$ is the matching that maximizes social welfare, but it is not stable as $m_2$ and $w_1$ both prefer each other more than the matched partner.

Stability $\nRightarrow$ max social welfare: In the setting above, $M' = \{(m_1, w_2), (m_2, w_1)\}$ is a stable pair, but it achieves less social welfare than $M$.