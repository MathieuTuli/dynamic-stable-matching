\section{Introduction}
 
The Stable Matching problem is a classical problem in computational social choice that attempts to find a stable matching between $N$ men and $N$ women. In the classical setting, each man $m_i$ has a strict, static preference ranking $\overrightarrow{\succ}_{m_i}$ for the women and vice versa, and the goal is to find a perfect and stable matching $M$ such that no pair of man $m$ and woman $w$ prefer each other to their current matches. In this work, we propose \textit{dynamic} stable matching, a variant where agents' rankings change over time, hence prompting a need to re-match agents according to the updated preference profiles. As such, a desirable matching algorithm should both factor in the agents' preferences, and produce long-lasting marriages across time.

% The Stable Matching problem is a classical problem in computational social choice that attempts to find a stable match between two equally sized opposing sets of agents, where each agent reports a vector of preferences over agents in the opposing set. In the traditional setting, preferences are static and the fundamental goal is to find a perfect, stable match: a match where each unique man and woman are paired such that no pair of man $m$ or woman $w$ prefer each other to their current match. In this work, we aim to investigate a variant of this stable matching problem, where agents' rankings are dynamic over time. Specifically, we investigate the more realistic scenario where agents have dynamic utility vectors over the set of opposing agents. These utility vectors may change at each time step, thus changing the underlying ranking of each agent over time.

We empirically evaluate various algorithms to analyze how the consistency and social welfare change of the generated matchings across time. We also perform an ablation study over various dynamics settings. Specifically, we investigate the following algorithms: (1) the classic Men/Women Preferred Deferred Acceptance (MPDA/WPDA); (2) a family of deterministic matching algorithms; and (3) a family of probabilistic matching algorithms. Our results show that while MPDA and WPDA generate stable matchings, they do not achieve as high of consistency and social welfare as other classes of algorithms.  