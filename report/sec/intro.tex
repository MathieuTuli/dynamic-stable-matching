\section{Introduction}
 
The Stable Matching (SM) problem is a classical problem in computational social choice that attempts to find a stable match between two equally sized opposing sets of agents, where each agent reports a vector of preferences over agents in the opposing set. In the traditional setting, preferences or rankings are static and the fundamental goal is to find a perfect, stable match: a match where each unique man and woman are paired such that no pair of man $m$ or woman $w$ prefer each other to their current match. In this work, we aim to investigate a variant of this stable matching problem, where agents' rankings are dynamic over time. Specifically, we investigate the more realistic scenario where agents have dynamic utility vectors over the set of opposing agents. Thus, these utility vectors may change at each time step, thus changing the underlying ranking of each agent over time.

Under this dynamic setting, we will investigate the use of various algorithms to analyze how consistency of matchings and social welfare change over time. Specifically, we investigate the following:
\begin{itemize}
    \item Men Preferred Deferred Acceptance (MPDA) under different dynamics of utilities. We will analyze how social welfare and consistency of matches varies as the agents' utilities are subject to various dynamics imposed by a Gaussian probability distribution
    \item
\end{itemize}